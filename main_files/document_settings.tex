

% Definicions generals del document


\documentclass[a4paper,12pt]{../style_files/my_thesis}
%\documentclass[a4paper,11pt]{book}
%\documentclass[a4paper,11pt]{report}

\usepackage{epsfig}
\usepackage{epstopdf}
\usepackage{graphics}
\usepackage{psfrag}
%\usepackage{color}
%\usepackage{times}
%\usepackage{newcent}
%\usepackage{palatino}
%\usepackage{mtimes}
\usepackage{../style_files/boxedminipage}
%\usepackage[ps2pdf,colorlinks]{hyperref}

% Paquets per al AMS-LaTeX
\usepackage[centertags,sumlimits,namelimits,reqno,fleqn]{amsmath}
\usepackage{amsfonts}
\usepackage{amssymb}
\usepackage{amsthm}
\usepackage{amscd}
\theoremstyle{plain} %plain, break, marginbreak, changebreak, change, margin (p. 253)

\newtheorem{theorem}{Theorem}[chapter]
\newtheorem{lemma}{Lemma}[chapter]
\newtheorem{corollary}{Corollary}[chapter]
\newtheorem{proposition}{Proposition}[chapter]
\theoremstyle{definition}
\newtheorem{definition}{Definition}[chapter]
\theoremstyle{remark}
\newtheorem{remark}{Remark}[chapter]
\theoremstyle{example}
\newtheorem{example}{Example}[chapter]
\theoremstyle{axiom}
\newtheorem{axiom}{Axiom}[chapter]


% Per a separar els paragrafs
\setlength{\parskip}{1mm}

% Per fer l'index conceptual
\usepackage{makeidx}

% Paquet per als headings (p.96-98)
\usepackage{../style_files/fancyheadings}
\pagestyle{fancyplain}



%\addtolength{\headwidth}{\marginparsep}  %surten a fora de la pagina?!?!
%\addtolength{\headwidth}{\marginparwidth}
\addtolength{\headwidth}{3pt}
\renewcommand{\chaptermark}[1]{\markboth{#1}{}}
\renewcommand{\sectionmark}[1]{\markright{\thesection\ #1}}
\lhead[\fancyplain{}{\bfseries\thepage}]{\fancyplain{}{\bfseries\rightmark}}
\rhead[\fancyplain{}{\bfseries\leftmark}]{\fancyplain{}{\bfseries\thepage}}
\cfoot{}

\newcommand{\clearemptydoublepage}{\newpage{\pagestyle{empty}\cleardoublepage}}


%per centrar les columnes que agafen 2 columnes
%\renewcommand{\multirowsetup}{\centering}

% Paquet per a requadres (p. 278)
%\usepackage{::style_files:fancybox}

% Paquet per a nous floats (p. 146-150)
%\usepackage{float}

% Paquet per a afegir espais al final de macros (newcommand) (p. 50)
\usepackage{xspace}
