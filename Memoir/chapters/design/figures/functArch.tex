
\begin{figure}[htpb]
	%\centering
	\hspace{-2cm}
\begin{tikzpicture}

	\tikzstyle{every node}=[rectangle,
		draw=black,fill=teal!20,
		minimum height=0.7cm,minimum width=2cm,
		align=center]
	\tikzstyle{edge from parent}=[black!50,thick,-o,draw]

	\tikzstyle{level 0}=[very thick]
	\tikzstyle{level 1}=[semithick,
		sibling distance=6cm,level distance=2cm,inner sep=3mm,
		edge from parent fork down]
	\tikzstyle{level 2}=[ultra thin,rounded corners,
		sibling distance=2cm,level distance=1.5cm,inner sep=2mm,
		grow=down,anchor=west,
		edge from parent path={(\tikzparentnode.south) |- (\tikzchildnode.west)},
		grow via three points={one child at (1.5em,-4em) and
		two children at (1.5em,-4em) and (1.5em,-7.5em)}]
	
	\node[fill=teal!70]	{Obstacle Collision Avoidance System}
		child	{node[fill=teal!40] {Communicate with\\ Ground Control}
			child	{node {Start OCAS}}
			child	{node {Log information}}
			child	{node {Stop OCAS}}
		}
		child	{node[fill=teal!40] {Detect obstacle}
			child	{node {Monitor environment}}
			child	{node {Confirm detection}}
			child	{node {Determine distance}}
			child	{node {Determine velocity}}
			child	{node {Establish level of threat}}
			child	{node {Decide on action}}
		}
		child	{node[fill=teal!40] {Avoid collision}
			child	{node {Compute trajectory}}
			child	{node {Compute\\ recquired actuation}}
			child	{node {Command actuation}}
		};


\end{tikzpicture}

	\caption{Functional architecture}
	\label{fig:functArch}
\end{figure}
