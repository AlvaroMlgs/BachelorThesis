
\begin{figure}[htbp]
	\centering

	\begin{tikzpicture}

		\tikzstyle{reference}=[rectangle,rounded corners=0.35cm,
			fill=black!30,draw=black,thin,
			minimum height=0.7cm,minimum width=2cm,inner sep=2mm,
			align=center,node distance=0.5cm]
		\tikzstyle{function}=[rectangle,rounded corners,
			fill=teal!40,draw=black,semithick,
			minimum height=0.7cm,minimum width=2cm,inner sep=2mm,
			align=center,node distance=0.5cm]
		\tikzstyle{logic}=[circle,
			fill=teal!70!black!70,text=white,
			inner sep=1mm,node distance=0.5cm]
		\tikzstyle{boundary}=[rounded corners,
			draw=red!70!black!100,very thick,dashed]
		\tikzstyle{arrow}=[>=stealth,->,thick]

		\node[function](function){Function};
		\node[right=0.5cm of function,align=left]{Represents an individual function or subfunction as defined in the Functional\\ Architecture from Figure \ref{fig:functArch}.};
		
		\node[logic,below=of function](logic){Logic};
		\node[right=0.9cm of logic,align=left]{Represents a logical \textit{and} or \textit{or} gate for defining parallel or alternative paths,\\ respectively.};
		
		\node[reference,below=of logic](reference){Reference};
		\node[right=0.5cm of reference,align=left]{Represents a reference block that specifies the origin or destination of a path\\ from an external function of the system.};

		\node[boundary,below=of reference,node distance=0.5cm](boundary){Boundary};
		\node[right=0.5cm of boundary,align=left]{Represents the boundaries of the functional description, be it the whole system\\ or a subfunction of it.};
		
		\draw[arrow] ($(boundary.center)+(-0.5,-1.2)$) -- ($(boundary.center)+(0.5,-1.2)$) ;
		\node[align=left,anchor=west] at ($(boundary.center)+(1.5,-1.2)$) {Indicates the sequential order that is to be followed from one function to another.};

		\draw[arrow,dashed] ($(boundary.center)+(-0.5,-2.2)$) -- ($(boundary.center)+(0.5,-2.2)$) ;
		\node[align=left,anchor=west] at ($(boundary.center)+(1.5,-2.2)$) {Indicates an information flow between two functional blocks.};


	\end{tikzpicture}

	\caption{Symbology used for the FFBDs}
	\label{fig:ffbdlegend}
\end{figure}
