
\let\textcircled=\pgftextcircled
\chapter{Testing and results} \label{chapter:testing}

\initial{I}n order to prove the effectiveness of the Obstacle Collision Avoidance System, it needs to be tested to prove that it meets the requirements and design specifications.
Hence, a set of individual tests have to be designed to assess the capabilities of each of the components involved in the system.

The present chapter will cover the experimental setups and results of those experiments performed on the critical components, subsystems and, finally, the system as a whole in a realistic environment.


\section{Testing methods}

Most of the testing was done in parallel to the implementation of the design into the real product, ensuring that the system was built over robust and properly working components, since possible modifications to the design are significantly more costly the later the development phase in which errors are detected (Figure \ref{fig:incose}).
Furthermore, tests can give valuable insight on the functioning of the components, which can prove useful in the decision making process within the implementation stages, effectively improving the overall performance of the system.

Thus, the actions described in this chapter can be seen as complementary and, in some sense also contributing, to the implementation phase of the OCAS.

\subsection{Component testing}

\subsubsection{Interfaces}

\subsubsection{Ultrasonic rangefinders}


\subsection{Software testing: SITL}


\subsection{System testing}


\section{Results}


