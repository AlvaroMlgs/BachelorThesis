
\let\textcircled=\pgftextcircled
\chapter{Conclusions} \label{chapter:conclusions}

The main goal of this thesis was to make unmanned flight more secure and flexible by means of a modular Obstacle Collision Avoidance System, as defined in Chapter \ref{chap:problem}. From the work done during the project, it can be concluded that:

\begin{itemize}

	\item There exists a large capacity of improvement in terms of the safety levels that could be achieved during the operation of Unmanned Aerial Vehicles, being the Obstacle Avoidance feature a mere example of the available potential.
		Advanced communications, ADS-B surveillance, decentralised swarming\ldots would also contribute to the common safety issue.

	\item There are several options in terms of environment sensing alternatives which provide various advantages and drawback to the OCAS in terms of range, accuracy or flexibility.
		The ultrasonic rangefinders have been chosen mainly for their ease of use and small processing requirements.

	\item The design phase of a complex system depends on a large amount of variables and assumptions, which are difficult to track if a systematic design procedure is not followed.
		For this project, as well as for the vast majority of systems developed by the most important companies, the Systems Engineering approach was selected for the orderly flow of ideas due to its proved reliability.

	\item The definition of interfaces is of utmost importance for the system to work as a whole and properly perform its functions.
		When two subsystems or components cannot communicate in a common language, an intermediate translation layer shall be created to translate the information.
		For example, between the UAV and the Python scripts, MAVproxy acted as a translator; or between the Python algorithms and MAVproxy, the DroneKit API translated the commands into usable MAVlink messages.

	\item Some decisions on the scripts architecture were made during the implementation and programming phases which nevertheless should have been made through the Systems Engineering methods applied to the design of each particular (software) subsystem.

	\item The theoretical or simulator-based inferences should be considered incomplete, and only blindly applicable to the scenarios that accurately represent the mathematical model being studied, as shown in Section \ref{sec:sitl}.

	\item Systematically testing the design and implementation of the product and its components is a very important step in the development process, allowing to validate both the solution and the process itself, and avoiding later modifications of the system that could be really expensive to resolve in the production or operation phases.

\end{itemize}

Finally, it is worth mentioning that Systems Engineering is a much more exhaustive activity than what was shown in this thesis, involving a large set of engineers and specialists working on projects that span several years of development and implementation.

\section{Summary of contributions}

Naturally, the main body of this thesis (Chapters \ref{chap:design}, \ref{chap:implementation} and \ref{chap:testing}) contains all the contributions that have been made to the field of UAVs during the execution of the project.
Nonetheless, they will be summarised here for easier access:

\begin{itemize}

	\item A conceptual safety layer has been designed to operate between the UAV and its physical surroundings.
		Such safety layer (the OCAS) gives additional information of the environment to the UAS so that more complex decisions can be made automatically, enhancing the situational awareness and overall safety of the vehicle.

		During the design process, a study has been made on the requirements that should be fulfilled by the Obstacle Collision Avoidance System, together with a logical decomposition of its functional and physical architectures, and the definition of the interfaces with the rest of the UAS.

	\item The proposed design has been implemented into a prototype as a proof of concept, focusing on the integration of the system and the creation of the interfaces.
		This approach was followed after the belief that once the successful integration of the system is complete, it will hardly need any modification, simplifying future research on the technical problems that are still to be solved (information capture, signal processing\ldots), while the robust baseline remains unchanged.

		Naturally, the proposed implementation is just one among many different alternatives, and the capabilities of the OCAS could be extended or enhanced with some modifications to the solution, while still meeting the requirements and the architecture of the system.

	\item Finally, a working prototype has been built according to the design guidelines, testing it in a series of realistic scenarios which show that the proposed solution is completely functional and fulfills the problem statement from Chapter \ref{chap:problem}, representing a good baseline on which to base further research.

\end{itemize}

\section{Future work}

The development of a complex system like the OCAS can by no means be complete within the duration of one single Bachelor's Thesis.
Thus, this project has been focused on providing a capable baseline on which to found future research.
In particular, some of the ideas that have been generated during the execution of the project but could not be further developed even though they can be interesting to work on in the future are:

\begin{itemize}

	\item The acquisition of data from the ultrasonic rangefinders can be greatly improved; from the location of the sensors for better spatial awareness to the triggering procedure, the noise filtering of the incoming signal to reduce false-positives or the processing of the data with more sophisticated algorithms to better predict a potential collision.

	\item The integration of alternative sensors, especially stereoscopic cameras but also radar or lidar, which can be more difficult to operate and extract useful information from than the sonar but also provide other advantages over it, as explained in Section \ref{sec:choice}.

	\item The improvement of the algorithms that predict the trajectory of the UAV with respect to the obstacle, to obtain a more reliable avoidance of potential collisions.

	\item The automation / simplification of the connection of the GCS with the OCAS and the script initialisation procedures.

	\item The improvement of the testing platform, since the available one has serious limitations in terms of the flight duration and the additional weight that the OCAS entails.

	\item The application of the UAV + OCAS system to real life problems, such as precise positioning in GPS neglected areas, indoor / urban navigation, localisation and mapping (SLAM), etc.

\end{itemize}
