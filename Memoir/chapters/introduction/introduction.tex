%
% File: introduction.tex
% Author: Victor F. Brena-Medina
% Description: Introduction chapter where the biology goes.
%
\let\textcircled=\pgftextcircled
\chapter{Introduction}

\initial{T}his chapter will be used to acquaint the reader with the emerging UAV market, and the challenges it is facing on its way towards maturity.
Also, the reasons for its rapid evolution will be exposed and finally, focusing on the contents of this thesis, the personal motivation and the methodology will be explaind to further expand on the topics of interes in the following chapters.

\section{Background information} 

The first remotely radio controlled models appeared in the early twentieth century as small prototypes for potential manned aircraft. 
Afterwards, and during most of the century, the investigation and development lines were directed towards the military scope, in which the main objective of UAVs, which is still applied today, was to substitute manned aircraft in three types of military operations, commonly known as ``the three D's" \cite{daily2015,uasapplications2016}:

\begin{itemize}
	\item Dirty: operations performed in a contaminated environment.
	\item Dangerous: operations entailing some risk for the pilot.
	\item Dull: long and monotone operations, such as monitoring operations.
\end{itemize}

In the 70's and the 80's, efforts were directed to improve the technical characteristics of these vehicles.
But it was not until the late 80's when a revolution in the industry took place with the introduction of the GPS navigation system, whose accuracy in geolocation opened a whole new spectrum of possibilities.

Regarding the civil sector, the potential applications of UAVs in the non-military field are much more diverse.
Nowadays these vehicles are in the process of finding new niche positions in the civilian market, having been introduced up to now in different industry sectors such as agriculture, forest fire fighting, search and rescue, aerial photography, cartography, or security and surveillance, among others. 
Despite the latter, the use of UAVs for civil purposes is relatively recent in comparison with the military sector.
This late implementation in the civilian field was caused mainly by two limitations which are of minor relevance in the fighting industry: legislation and economy.
\cite{aguado2016}

\section{Legal framework}

\cite{manualonremotelypilotedaircraftsystemsrpas2015,ley182014de15deoctubredeaprobaciondemedidasurgentesparaelcrecimientolacompetitividadylaeficiencia2014}

\section{Socioeconomic environment}

\section{Motivation}

Traditionally, the most important payload that could be carried in an aircraft was human beings, that would perfom their mission while aloft.
Nevertheless, the advancements on sensing technology and wireless communications have forced a change on traditional aviation.
Apart from commercial aviation, where the final objective is to transport people form one place to another, in almost any other mission the role of the human workforce is to pilot the aircraft and/or operate the payload systems.
This secondary role of the human operators implies that, given the maturity of the involved technology, they could be substituted by intelligent computer systems or, at least, disembarked form the aircraft into a safer Ground Control Station.
The process of ``unmanning'' the aircraft also brings the advantages of decreasing the weight of the aircraft and thus improving its endurance and manoeuvrability, avoids putting the pilot in a dangerous situation, and helps alleviate the errors associated with tedious and repetitive tasks, among others.

However, there are also some downsides.
The most accused ones for experienced pilots are those related with the loss of situational awareness that comes as a result of eliminating the pysical cues (body inertia, vibrations\ldots) and relying on instrumental readings only, and also the limited number of control input, which for civil UAVs do not exceed 8 or 9 scalar channels.

\section{Project objectives}

\section{Methodology}

\section{Time planning}

