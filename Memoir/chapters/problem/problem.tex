
\let\textcircled=\pgftextcircled
\chapter{Problem statement} \label{chap:problem}

\initial{T}he objectives of the project have already been stated in Section \ref{sec:objectives}.
Following those ideas, the problem that is to be answered in this thesis can be stated with the following question:

\begin{flushright}
	\fboxsep0pt
	\colorbox{teal!30}{\begin{minipage}[c]{0.01\textwidth}
		~\\[6em]
	\end{minipage}}
	\hspace{-2em}
	\begin{minipage}[c]{0.95\textwidth}
		\begin{quote}
			\large \em
			Is it possible to improve the operational safety of a wide range of UAVs by developing an intermediate functional layer that prevents physical collisions between the UAV and its surroundings?
		\end{quote}
	\end{minipage}
\end{flushright}

The above statement tries to condense the main idea of the thesis in a compact and precise manner.
Nonetheless, some concepts within it might need some clarification:

\begin{description}

	\item[\itshape Operational safety:] A reliable collision avoidance system reduces the workload of the pilot so that higher-level tasks directly related to the operation of the mission can be performed more efficiently and safely.
	\item[\itshape Wide range of UAVs:] As stated in Chapter \ref{chap:ardupilot} the project will focus on the widely-spread Ardupilot firmware, which is currently the leading alternative of open-source UAV controller software available.
	\item[\itshape Intermediate functional layer:] The proposed solution shall be easily integrable within existing UAVs; offered as an enhancement to the toolbox of functions of the system.
		Thus, the solution shall incorporate additional features to the UAS, while the functions provided by Ardupilot should remain unaffected.

\end{description}

The problems is believed to be worthwhile answering since, as it was proven in the State of the Art study (Chapter \ref{chap:sota}), the technology is not mature and has not been implemented except on very specific products like the DJI Phantom 4.

Furthermore, a fully operable and reliable OCAS would allow for a more autonomous operation of the UAV, avoiding the pilot from needing to be focused on the immediate surroundings of the vehicle, thus permitting for an improved situational awareness and leading to a better overall execution of the mission.
Ultimately, a higher safety level of general UAV operations could lead the authorities to reconsider the possibility of them flying for civil purposes with less restrictions, thus enabling companies to save vast amounts of money and manpower on the execution of activities that are currently being accomplished in a less effective way by human workers.


