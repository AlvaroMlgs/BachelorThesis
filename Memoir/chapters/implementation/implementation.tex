
\let\textcircled=\pgftextcircled
\chapter{System implementation}	\label{chap:implementation}

\initial{T}he design phase of the Systems Engineering approach will be presented in the current chapter instead of the previous one since the focus of this phase was put in the software part, for the hardware one (mainly structural mounts) being too dependant on the configuration of the existing UAV.
Hence, all the design, implementation and testing of the software branch were conducted in a parallel manner, as will be exposed in this part of the thesis.
Nevertheless, the final hardware assembly of the system in the working prototype will also be discussed at the end of the present chapter.
%Also, the software design was a parallel process to the coding itself, and the functioning of the computer programs is better explained through real code snippets rather than pseudocode or block diagrams, although some might be used.

Chapter \ref{chap:implementation} will describe the complete implementation of the Obstacle Collision Avoidance System within the Unmanned Aerial System starting from the uppermost level and deepening through the execution of the interfaces and the software layers down to the custom-built control script.


\section{The OCAS within the UAS}

This section describes the architecture of the UAS prior to the implementation of the OCAS.
Then, the uppermost integration level is explained, emphasising the compliance with Requirement 3.4 (lack of interference with the Ardupilot functions)

\subsection{Overview of the existing UAS}

The regular Ardupilot-based Unmanned Aerial System with Ground Control Station capabilities is composed of three main subsystems:

Firstly, the UAV, which is considered to be fully operable.
That is, the UAV concept encloses the airframe, propulsion, power source and all the other components as described in \cite{arteta2015}; but most importantly, the controller board with the Ardupilot software, considered as the ``brain'' of the UAV.

Secondly, the pilot with the Radio Control transmitter (see Figure \ref{fig:RCtransmitter}) will also be considered a subsystem of the UAS.
He/she has direct control of the UAV when flying in manual mode, plus is responsible of the operation of the GCS when the UAV is in Automatic mode (see Chapter \ref{chap:ardupilot}).

Thirdly, the computer running the GCS software and having a real-time wireless connection with the UAV while in the air.

\begin{figure}[htbp]
	\centering
	\begin{tikzpicture}
		
	\end{tikzpicture}
	\caption{Regular Ardupilot UAS architecture}
	\label{fig:uas}
\end{figure}


In addition, the interfaces between these subsystems are depicted in Figure \ref{fig:uas} and work as follows:

The Radio Control (RC) link is established between the RC transmitter held by the pilot and the RC receiver that is directly connected to the controller board.
A 2.4 GHz signal transmits information on the position of the control sticks as a PWM directly to the Ardupilot software, as explained in Section \ref{sec:basics}.

Likewise, the telemetry link consists of a 433 MHz duplex radio wave that carries MAVlink messages from the UAV to the GCS and viceversa, allowing for configuration, calibration and operation of the autonomous flight modes while the vehicle is aloft.

\subsection{Integration of the OCAS} \label{sec:ocasintegration}

The introduction of the OCAS into the UAS shall preserve the basic Ardupilot functions.
Thus, the architecture should not be significantly modifyied.
The final decision on the UAS architecture after the integration of the OCAS is shown in Figure \ref{fig:uas-ocas}

\begin{figure}[htbp]
	\centering
	\begin{tikzpicture}
		
		\newcommand{\drawpilot}[1]{
			\node[rounded corners=2pt,minimum height=1.3cm,minimum width=0.4cm,fill=gray,#1] (body) {};
			\node[circle,fill=gray,minimum size=5mm,above=1pt of body] (head) {};
			\draw[line width=1mm,round cap-round cap,gray] ([shift={(2pt,-1pt)}]body.north east) --++(-90:6mm);
			\draw[line width=1mm,round cap-round cap,gray] ([shift={(-2pt,-1pt)}]body.north west)--++(-90:6mm);
			\draw[thick,white,-round cap] (body.south) --++(90:5.5mm);
			\node[below of=pilot](pilot-label){Pilot};

			\node[fit=(body)(head)(pilot-label),
				inner sep=3mm,rounded corners=5mm]
				(pilot) {}; %Group the pilot together
		}

		\newcommand{\drawgcs}[1]{
			\node[rectangle,rounded corners,double,draw=gray,very thick,minimum height=1cm,minimum width=1.5cm,#1](display){GCS};
			\node[trapezium,rounded corners,draw=gray,ultra thick,trapezium angle=50,minimum height=0.3cm,inner ysep=1pt,inner xsep=0.15cm,below=2pt of display](keyboard){};

			\node[fit=(display)(keyboard),
				inner sep=5mm,rounded corners=5mm] 
				(gcs) {};	%Group the GCS together
		}

		\newcommand{\drawuav}[1]{
			\node[#1](coord){};
			\coordinate[fill=gray,draw=gray,circle,minimum size=0.15cm,inner sep=0] (m1) at ($(coord)+(-0.7cm,0.7cm)$) {};
			\coordinate[fill=gray,draw=gray,circle,minimum size=0.15cm,inner sep=0] (m2) at ($(coord)+(0.7cm,0.7cm)$) {};
			\coordinate[fill=gray,draw=gray,circle,minimum size=0.15cm,inner sep=0] (m3) at ($(coord)+(0.7cm,-0.7cm)$) {};
			\coordinate[fill=gray,draw=gray,circle,minimum size=0.15cm,inner sep=0] (m4) at ($(coord)+(-0.7cm,-0.7cm)$) {};

			%Airframe
			\path[fill=gray,draw=gray,line width=1mm] (m1.base)
				.. controls (coord) .. (m2.base)
				.. controls (coord) .. (m3.base)
				.. controls (coord) .. (m4.base)
				.. controls (coord) .. (m1.base);
				

			%Propellers
			\path[fill=gray,draw=white] (m1) arc (155:115:0.8cm) arc (335:295:0.8cm)
				arc (115:155:0.8cm) arc (295:335:0.8cm);
			\path[fill=gray,draw=white] (m2) arc (65:25:0.8cm) arc (245:205:0.8cm)
				arc (25:65:0.8cm) arc (205:245:0.8cm);
			\path[fill=gray,draw=white] (m3) arc (155:115:0.8cm) arc (335:295:0.8cm)
				arc (115:155:0.8cm) arc (295:335:0.8cm);
			\path[fill=gray,draw=white] (m4) arc (65:25:0.8cm) arc (245:205:0.8cm)
				arc (25:65:0.8cm) arc (205:245:0.8cm);

			\node[yshift=-0.7cm](uav-label) at (coord) {UAV};

			\node[fit=(coord)(m1)(m2)(m3)(m4)(uav-label),
				inner sep=5mm,rounded corners=5mm]
				(uav) {};	%Group the UAV together
		}

		\newcommand{\drawocas}[1]{
			\node[#1,fill=teal,minimum height=2cm,minimum width=3cm](ocas-block){};
			\coordinate[label=above right:\textcolor{white}{\bfseries OCAS}]
				(ocas-label) at (ocas-block.south west) {};

				\node[fit=(ocas-block)(ocas-label)](ocas) {};
		}


		\drawpilot{}
		\drawgcs{below right=2cm of pilot,xshift=2cm}
		\drawuav{below left=2.7cm of pilot,xshift=-2cm}
		\drawocas{below =5cm of pilot}

		\draw[loosely dashdotted,thick] (pilot) -- node[above,sloped]{RC} (uav);
		\draw[thick] (pilot) -- node[above,sloped]{Physical} (gcs);
		\draw[dashed,thick] (uav) -- node[above,sloped]{Telemetry} (gcs);
		\draw[thick,decorate,decoration={coil,aspect=0,segment length=1cm}] (ocas) -- node[below=2mm,sloped]{MAVlink} (uav);
		\draw[thick,decorate,decoration=zigzag] (ocas) -- node[below=2mm,sloped]{SSH (WiFi)} (gcs);

	\end{tikzpicture}

	\caption{OCAS-equiped UAS architecture}
	\label{fig:uas-ocas}
\end{figure}


With this setup, the original connections and functions are maintained while the OCAS, which is mounted onboard the UAV, communicates with it through a USB cable via the MAVlink protocol (the same one used for the telemetry link).
Additionally the GCS has a second wireless link to the OCAS via WiFi, making use of the SSH (plus optional X Window System forwarding) protocol.
More information on these interfaces is provided in Section \ref{sec:peripheral}.

\section{OCAS peripheral connections (hardware interfaces)} \label{sec:peripheral}

As already stated in Section \ref{sec:interfaces}, the main component of the OCAS is the computer board, which can be considered as a hub on which the rest of the components of the OCAS are brought together.
Thus, the first step is to define the information pathways of the Raspberry Pi with the oter hardware components.

The physical layout of the OCAS is shown in Figure \ref{fig:ocas}. 
Notice that only Raspberry Pi peripherals are being considered. 
They are handled in the following manner:


\begin{figure}[htbp]
	\centering
	\begin{tikzpicture}

		\newcommand{\drawgcs}[1]{
			\node[rectangle,rounded corners,double,draw=gray,very thick,minimum height=1cm,minimum width=1.5cm,#1](display){GCS};
			\node[trapezium,rounded corners,draw=gray,ultra thick,trapezium angle=50,minimum height=0.3cm,inner ysep=1pt,inner xsep=0.15cm,below=2pt of display](keyboard){};

			\node[fit=(display)(keyboard),
				inner sep=5mm,rounded corners=5mm] 
				(gcs) {};	%Group the GCS together
		}

		\newcommand{\drawuav}[1]{
			\node[#1](coord){};
			\coordinate[fill=gray,draw=gray,circle,minimum size=0.15cm,inner sep=0] (m1) at ($(coord)+(-0.7cm,0.7cm)$) {};
			\coordinate[fill=gray,draw=gray,circle,minimum size=0.15cm,inner sep=0] (m2) at ($(coord)+(0.7cm,0.7cm)$) {};
			\coordinate[fill=gray,draw=gray,circle,minimum size=0.15cm,inner sep=0] (m3) at ($(coord)+(0.7cm,-0.7cm)$) {};
			\coordinate[fill=gray,draw=gray,circle,minimum size=0.15cm,inner sep=0] (m4) at ($(coord)+(-0.7cm,-0.7cm)$) {};

			%Airframe
			\path[fill=gray,draw=gray,line width=1mm] (m1.base)
				.. controls (coord) .. (m2.base)
				.. controls (coord) .. (m3.base)
				.. controls (coord) .. (m4.base)
				.. controls (coord) .. (m1.base);
				

			%Propellers
			\path[fill=gray,draw=white] (m1) arc (155:115:0.8cm) arc (335:295:0.8cm)
				arc (115:155:0.8cm) arc (295:335:0.8cm);
			\path[fill=gray,draw=white] (m2) arc (65:25:0.8cm) arc (245:205:0.8cm)
				arc (25:65:0.8cm) arc (205:245:0.8cm);
			\path[fill=gray,draw=white] (m3) arc (155:115:0.8cm) arc (335:295:0.8cm)
				arc (115:155:0.8cm) arc (295:335:0.8cm);
			\path[fill=gray,draw=white] (m4) arc (65:25:0.8cm) arc (245:205:0.8cm)
				arc (25:65:0.8cm) arc (205:245:0.8cm);

			\node[yshift=-0.7cm](uav-label) at (coord) {UAV};

			\node[fit=(coord)(m1)(m2)(m3)(m4)(uav-label),
				inner sep=5mm,rounded corners=5mm]
				(uav) {};	%Group the UAV together
		}

		\newcommand{\drawocas}[1]{
			\node[#1,fill=teal,opacity=0.4,minimum height=5cm,minimum width=7cm](ocas-block){};
			\coordinate[label=above right:\textcolor{white}{\bfseries OCAS}]
				(ocas-label) at (ocas-block.south west) {};

				\node[fit=(ocas-block)(ocas-label)](ocas) {};
		}

		\newcommand{\drawrpi}[1]{
			\node[#1,fill=green!70!black,minimum height=2.5cm,minimum width=4cm,rounded corners=2mm](rpi) {};
			\coordinate[label=above right:\textcolor{white}{RPi}]
				(rpi-label) at (rpi.south west) {};
			\node[draw,fill=gray!50,minimum height=6mm,minimum width=8mm,anchor=north west,yshift=-2pt](usb1) at (rpi.north west) {\includegraphics[width=5mm]{./figures/usb.png}};
			\node[draw,fill=gray!50,minimum height=6mm,minimum width=8mm,below=2pt of usb1](usb2) {\includegraphics[width=5mm]{./figures/usb.png}};
			\node[draw,fill=gray!50,minimum height=3mm,minimum width=4mm,anchor=north east,xshift=-0.7cm](microusb) at (rpi.north east) {};
			\node[fill=black!80,minimum height=3mm,minimum width=20mm,anchor=south east,xshift=-2mm,yshift=2mm](gpio) at (rpi.south east) {};
				\node[right=0.1cm of gpio]{\footnotesize \scshape gpio};
			\foreach \position in {1,2,...,15}{
				\node[fill=yellow!60,circle,inner sep=0.5pt](pin\position) at ($(gpio.north west)+(\position/8,-1mm)$) {};
				\node[fill=yellow!60,circle,inner sep=0.5pt](pin\position+15) at ($(gpio.north west)+(\position/8,-2mm)$) {};
			}
		}

		\drawocas{}
		\drawuav{left=2.5cm of ocas,yshift=0.3cm}
		\drawgcs{left=2cm of ocas,yshift=3.5cm}
		\drawrpi{}

		% Wifi
		\node[fill,inner sep=0pt,minimum width=1mm,minimum height=6mm,above left=0.5cm of usb1](wifi-stick){};
		\node[circle,inner sep=0.7mm,fill,above=-1pt of wifi-stick](wifi-ball) {};
		\node[fit=(wifi-stick)(wifi-ball)](wifi) {};

		% Power
		\node[fill=black!50,minimum height=4mm,minimum width=7mm,above right=0.5cm of microusb](power) {\footnotesize Power};

		% Sonar
		\node[fill=blue!70,minimum height=5mm,minimum width=9mm,below=0.7cm of gpio,xshift=5mm](sonar-board){};
		\node[fill=black!80,draw=black!50,very thick,circle,inner sep=1mm,xshift=-2mm](sonar1) at (sonar-board) {};
		\node[fill=black!80,draw=black!50,very thick,circle,inner sep=1mm,xshift=2mm](sonar2) at (sonar-board) {};
		\node[fit=(sonar-board)(sonar1)(sonar2)](sonar){};


		% Connections (paths)
		\draw[semithick,decorate,decoration={coil,aspect=0,segment length=1.1cm}] (usb2.west) -- node[below=2mm,sloped]{MAVlink} (uav);
		\draw[semithick,decorate,decoration={expanding waves,angle=4,segment length=2mm}] (wifi-ball.north east) -- node[above=2mm,sloped]{SSH (WiFi)} (gcs);
		\draw[semithick] (usb1.west) -| (wifi-stick.south);
		\draw[semithick,double] (microusb) |- (power);
		\draw[semithick,decorate,decoration=zigzag] (sonar-board) -- (sonar-board|-gpio.south);
		%\foreach \wire in {-1.5,-0.5,0.5,1.5}
		%	\draw ($(sonar-board.north)+(\wire/20,0)$) -- ($(sonar-board.north|-gpio.south)+(\wire/20,0)$);

	\end{tikzpicture}

	\caption{OCAS hardware layout}
	\label{fig:ocas}
\end{figure}


\subsection{Power connection}
For the Raspberry Pi to boot up, the only requirement is to provide a continuous current of 5V and enough current to power any other peripheral as well as the board, which in any case will not be higher than 2A.
Thus, the battery pack, providing a continuous source of energy during during the whole duration of the mission is enough to meet the requirement.
It is connected to the Raspberry Pi SBC via a conventional USB type A, at the battery end, to micro-USB type B, at the computer end; no additional action being required.

\subsection{MAVlink connection}
The connection with the UAV (i.e. with the Ardupilot controller board) is done also via a regular USB cable, making use of the serial communications protocol.
The serial protocol is a simple manner of tranfering information which consists on transmitting the data one bit at a time, avoiding the synchronisation problem.
Hence, the only issue is that both ends must agree in advance on the transmission rate.
This is done by setting a common ``baud rate'', where a baud is the unit for symbol change (signal event) rate, commonly measured in bits per second.
In the particular case of communicating with the UAV, the messages transmitted through the serial link are defined according to the MAVlink protocol.

\subsection{GCS connection}
The link with the Ground Control Station is composed of two intermediate steps:

On one hand, the network adapter is connected to a USB port on the Raspberry Pi to provide the SBC with wireless networking capabilities.
This connection is entirely handled by the kernel, the adapter's drivers and the operating system, and needs no further action from the engineer.

The second step is decidedly more complex.
Firstly, the Raspberry Pi needs to be set up to wirelessly connect to the same network as the GCS computer.
There are several ways to achieve this goal, but an uncomplicated one is to create an ad-hoc network from the GCS computer (running Windows) to which the Raspberry Pi is directly connected.
The specific details are explained in Appendix \ref{app:network}.
This approach has been mainly chosen for its simplicity and portability, but notice that there exist more advanced network architectures that could provide significantly better performance.
Secondly, the SSH connection needs to be stablished over the network.
The process involves searching for the Raspberry Pi's address on the network, connecting to the SSH port and, optionally, setting up an X server for an easier Graphical User Interface (GUI) with the OCAS.
More details on the steps to be taken are developed in Appendix \ref{app:ssh}.

\subsection{GPIO connection}

The General Purpose Input / Output pins on the Raspberry Pi operate on a notably lower level than the previous hardware connections.
As their name implies, the GPIO pins are the most general type of connection the Raspberry Pi can handle.
The reason is that these pins have to be manually operated; that is, each of the pins can be set via software to either a HIGH or LOW state, meaning 3.3V or 0V with respect to the Ground (GND) potential, respectively.

Hence, in this project, the GPIO pins will be used to both trigger the ultrasonic rangefinders and read the returning signal that encodes the information on the distance from the sensor to the detected obstacle.

Besides, the sonar is equipped with its own microcontroller, which handles the lowest-level signals.
For its operation, it counts with 4 different pins (see Figure \ref{fig:hc-sr04}) 
\begin{enumerate}
	\item GND, or Ground, specifies the reference voltage of the device.
	\item VCC, which stands for Voltage Continuous Current, powers the sensor at 5V.
	\item Trigger is an input signal pin.
		A HIGH value on this pin triggers (hence the name) a series of short bursts of sound from the piezoelectric speaker, which will rebound on any close obstacle.
	\item Echo is the output signal pin.
		The sensor's microcontroller processes the sound captured by the microphone and sends a short pulse through the echo pin exactly after the rebounded sound signal is received.
		Knowing the speed of propagation of sound (given by $a=\sqrt{\gamma R_g T}$) and the time taken for the wave to travel to the obstacle and back, the distance can be calculated with $d=v \cdot t/2$
\end{enumerate}
The ultrasonic rangefinder's technical documentation can be found in Appendix \ref{app:sonar}.

On the Raspberry Pi side, the VCC pin shall be connected to any 5V pin, the GND pin to a Ground pin, and the Trigger and Echo pins to any numbered GPIO pins, depicted in Figure \ref{fig:gpio}.


\begin{figure}[htbp]
	\centering
	\includegraphics[width=0.55\textwidth]{./figures/gpio.png}
	\caption{GPIO pins on the Raspberry Pi 2 model B {\footnotesize Source: \url{raspberrypi.org}}}
	\label{fig:gpio}
\end{figure}


There is one important issue that needs to be noticed, though.
The rangefinders work on 5V only and, while the Raspberry Pi can provide 5V to power the sensors, the GPIO pins can be damaged if operated at more than 3.3V.
Thus, the signal pins must be reduced from 5V to 3.3V before being connected to the SBC.
The solution to the problem is to use a ``voltage divider'', which is a passive circuit that outputs a fraction of the input voltage by means of a pair of resistors, which are connected as shown in Figure \ref{fig:voltage}.

\begin{figure}[htbp]
	\centering
	\begin{tikzpicture}[circuit ee IEC,large circuit symbols]

		%\node[ground,point down,info={GND}](gnd){};	
		%\node[resistor,point up,info={R$_1$},right of=gnd](r1){};
		%\coordinate[above of=r1](coord);
		%\node[resistor,point up,info={R$_2$},right of=coord](r2){};
		%\node[contact,info={V$_{in}$},above left of=r2,yshift=0.4cm](vin){};
		%\node[contact,info={V$_{out}$},right of=coord](vout){};

		%\draw (vin) -| (r2) to (coord) to (vout) to (coord) to (r1) to (gnd);

		\node[ground,info=GND](gnd){};	
        \node[resistor,info=R$_2$,left=1 of gnd](r2){};
        \coordinate[left of=r2](coord);
        \node[resistor,info=R$_1$,left of=coord](r1){};
        \node[contact,info=left:V$_{in}$,above left of=r1,xshift=-10](vin){};
		\node[contact,info=left:V$_{out}$,below of=coord](vout){};

		\draw (vin) |- (r1) to (coord) to (vout) to (coord) to (r2) to (gnd);
		
	\end{tikzpicture}
	\caption{Schematic of a voltage divider}
	\label{fig:voltage}
\end{figure}


In the present case, for the voltage to drop from 5V to 3.3V, the resistors need to meet:
\begin{equation}
	\frac{V_{in}}{R_1+R_2}=\frac{V_{out}}{R_2} \ \Rightarrow\ \frac{V_{in}}{V_{out}}=\frac{R_1}{R_2}+1 \ \Rightarrow\ \frac{R_1}{R_2}=\frac{5V}{3.3V}-1 \ \Rightarrow\ \frac{R_1}{R_2}=\frac{1}{2} 
\end{equation}

So finally, an ultrasonic rangefinder connected to GPIO pins 14 and 15, for instance, would be connected a shown in Figure \ref{fig:sonarconnect}.

\begin{figure}[htbp]
	\centering
	\begin{tikzpicture}

		\node(rpi){\includegraphics[width=0.5\textwidth,trim={0 6cm 0 0},clip]{./figures/gpio.png}};
		\node[above=15mm of rpi,xshift=-1.2cm](sonar){\includegraphics[width=0.3\textwidth,trim={0 2mm 0 6mm},clip]{./figures/hc-sr04.jpg}};

		\coordinate(VCC) at ($(sonar.south)+(-3.8mm,5mm)$);
		\coordinate(TRIG) at ($(sonar.south)+(-1.5mm,5mm)$);
		\coordinate(ECHO) at ($(sonar.south)+(1mm,5mm)$);
		\coordinate(GND) at ($(sonar.south)+(3.3mm,5mm)$);

		\coordinate(v) at ($(sonar.south)+(-3.8mm,0)$);
		\coordinate(t) at ($(sonar.south)+(-1.5mm,-2mm)$);
		\coordinate(e) at ($(sonar.south)+(1mm,-7mm)$);
		\coordinate(g) at ($(sonar.south)+(3.3mm,-15mm)$);

		\coordinate(vcc) at ($(rpi.north west)+(17.6mm,-5.7mm)$);
		\coordinate(gnd) at ($(rpi.north west)+(20.8mm,-5.7mm)$);
		\coordinate(trig) at ($(rpi.north west)+(24.0mm,-5mm)$);
		\coordinate(echo) at ($(rpi.north west)+(27.0mm,-5mm)$);

		\node[draw,fill=white,minimum width=2mm,minimum height=5mm,inner sep=0,above=2mm of e](r1){\tiny R$_1$};
		\node[draw,fill=white,minimum width=2mm,minimum height=5mm,inner sep=0,above=7mm of gnd](r2){\tiny R$_2$};

		\draw[line width=0.5mm,blue!50] (e) -| (r2);
		\draw[line width=0.5mm,blue!50] (r2) -- (gnd);

		\draw[line width=0.5mm,red] (VCC) -- (v) -| (vcc);
		\draw[line width=0.5mm,black] (GND) -- (g) -| (gnd);
		\draw[line width=0.5mm,green!80] (TRIG) -- (t) -| (trig);
		\draw[line width=0.5mm,blue!80] (ECHO) -- (r1);
		\draw[line width=0.5mm,blue!80] (r1) -- (e) -| (echo);
		
		\node[fill=blue!80,circle,inner sep=0.4mm,above=12.3mm of echo] (connEcho){};
		\node[fill=black,circle,inner sep=0.4mm,above=5mm of gnd] (connGnd){};

	\end{tikzpicture}
	\caption{Connection of the HC-SR04 sensor to the Raspberry Pi}
	\label{fig:sonarconnect}
\end{figure}



\section{Software: Bringing everything together}

Having several flows of information arriving to the Raspberry Pi, it is crucial to set up a system that acquires all the data before it can be processed. 
In the present section such a system will be described.

\subsection{The Operating System}

The first software layer on the Raspberry Pi (apart from the kernel) is the Operating System (OS).
In this case, the linux OS is Raspbian Wheezy, which is a version of Debian adapted to be run on the Raspberry Pi's ARMv7 chip.

Raspbian is a complete OS, and as such its abilities are varied, being the most relevant for the project the network management tools and the capability of running external software applications.
Within the OCAS, Raspbian will be used as container of the software subsystems specified in the Logical Decomposition phase of the design process (Figure \ref{fig:pbs}), plus it will directly handle the functions associated to the GCS connector.
An schematic of the relevant software architecture to the OCAS is represented in Figure \ref{fig:software}.

\begin{figure}[htbp]
	\centering
	\begin{tikzpicture}

		\newcommand{\drawrpi}[1]{
			\node[#1,fill=green!70!black,minimum height=6cm,minimum width=10cm,rounded corners=2mm](rpi) {};
			\coordinate[label=above right:\textcolor{white}{RPi}]
				(rpi-label) at (rpi.south west) {};
			\node[draw,fill=gray!50,minimum height=6mm,minimum width=8mm,anchor=north west,yshift=-2pt](usb1) at (rpi.north west) {\includegraphics[width=5mm]{./figures/usb.png}};
			\node[draw,fill=gray!50,minimum height=6mm,minimum width=8mm,below=2pt of usb1](usb2) {\includegraphics[width=5mm]{./figures/usb.png}};
			\node[draw,fill=gray!50,minimum height=3mm,minimum width=4mm,anchor=north east,xshift=-0.7cm](microusb) at (rpi.north east) {};
			\node[fill=black!80,minimum height=3mm,minimum width=20mm,anchor=south east,xshift=-2mm,yshift=2mm](gpio) at (rpi.south east) {};
				%\node[right=0.1cm of gpio]{\footnotesize \scshape gpio};
			\foreach \position in {1,2,...,15}{
				\node[fill=yellow!60,circle,inner sep=0.5pt](pin\position) at ($(gpio.north west)+(\position/8,-1mm)$) {};
				\node[fill=yellow!60,circle,inner sep=0.5pt](pin\position+15) at ($(gpio.north west)+(\position/8,-2mm)$) {};
			}
		}

		\newcommand{\drawgcs}[1]{
			\node[rectangle,rounded corners=3pt,double,draw=gray,thick,minimum height=0.3cm,minimum width=0.5cm,#1](display){\footnotesize GCS};
			\node[trapezium,rounded corners=2pt,draw=gray,very thick,trapezium angle=50,minimum height=0.1cm,inner ysep=1pt,inner xsep=3mm,below=2pt of display](keyboard){};

			\node[fit=(display)(keyboard),inner sep=0mm] 
				(gcs) {};	%Group the GCS together
		}

		\newcommand{\drawuav}[1]{
			\node[#1](coord){};
			\coordinate[fill=gray,draw=gray,circle,minimum size=0.1cm,inner sep=0] (m1) at ($(coord)+(-0.4cm,0.4cm)$) {};
			\coordinate[fill=gray,draw=gray,circle,minimum size=0.1cm,inner sep=0] (m2) at ($(coord)+(0.4cm,0.4cm)$) {};
			\coordinate[fill=gray,draw=gray,circle,minimum size=0.1cm,inner sep=0] (m3) at ($(coord)+(0.4cm,-0.4cm)$) {};
			\coordinate[fill=gray,draw=gray,circle,minimum size=0.1cm,inner sep=0] (m4) at ($(coord)+(-0.4cm,-0.4cm)$) {};

			%Airframe
			\path[fill=gray,draw=gray,line width=0.7mm] (m1.center)
				.. controls (coord) .. (m2.center)
				.. controls (coord) .. (m3.center)
				.. controls (coord) .. (m4.center)
				.. controls (coord) .. (m1.center);
				

			%Propellers
			\path[fill=gray] (m1) arc (155:115:0.5cm) arc (335:295:0.5cm)
				arc (115:155:0.5cm) arc (295:335:0.5cm);
			\path[fill=gray] (m2) arc (65:25:0.5cm) arc (245:205:0.5cm)
				arc (25:65:0.5cm) arc (205:245:0.5cm);
			\path[fill=gray] (m3) arc (155:115:0.5cm) arc (335:295:0.5cm)
				arc (115:155:0.5cm) arc (295:335:0.5cm);
			\path[fill=gray] (m4) arc (65:25:0.5cm) arc (245:205:0.5cm)
				arc (25:65:0.5cm) arc (205:245:0.5cm);

			\node[yshift=-0.8cm](uav-label) at (coord) {\footnotesize UAV};

			\node[fit=(coord)(m1)(m2)(m3)(m4)(uav-label),inner sep=1.4mm]
				(uav) {};	%Group the UAV together
		}

		\newcommand{\drawsensor}[1]{
			\node[#1,fill=blue!70,minimum height=5mm,minimum width=9mm](sonar-board){};
			\node[fill=black!80,draw=black!50,very thick,circle,inner sep=1mm,xshift=-2mm](sonar1) at (sonar-board) {};
			\node[fill=black!80,draw=black!50,very thick,circle,inner sep=1mm,xshift=2mm](sonar2) at (sonar-board) {};
			\node[fit=(sonar-board)(sonar1)(sonar2),inner sep=0](sonar){};
		}

		\newcommand{\drawos}[1]{
			\node[#1,fill=brown,minimum height=4.5cm,minimum width=7cm](os){};
			\coordinate[label=above right:\textcolor{white}{Raspbian}]
				(raspbian-label) at (os.south west) {};
			\node[fill=brown!80!black,anchor=north west,align=left,inner sep=2mm](network) at (os.north west) {Network tools\\ SSH client};
			%\node[rectangle,right=1cm of network](shell) {Unix shell};
			\node[fill=yellow!90!black,rounded corners,anchor=south east,minimum height=2.7cm,minimum width=4.6cm](python) at ($(os.south east)+(-3mm,3mm)$) {};
			\coordinate[label=above right:\textcolor{white}{Python}]
				(python-label) at (python.south west) {};
			\node[anchor=north west](gui) at ($(python.north west)+(6mm,-2mm)$) {GUI};
			\node[anchor=west](mavproxy) at ($(python.west)+(1mm,-2mm)$) {MAVproxy};
			\node[anchor=north east,align=right](script) at ($(python.north east)+(-3mm,-2mm)$) {Control\\ script};
		}


		\begin{scope}[transparency group,opacity=0.4]
			\drawrpi{}
			\drawgcs{left=of usb1,yshift=1mm}
			\drawuav{below left=of usb2,xshift=-3mm}
			\drawsensor{right=of gpio}
		\end{scope}
		\drawos{}

		\coordinate (mav) at ($(mavproxy)+(-3.5cm,0)$);

		\draw[thick,decorate,decoration={expanding waves,angle=10,segment length=0.8mm}] (usb1) -- node[above=2mm,sloped]{SSH} (gcs);
		\draw[thick,rounded corners] (usb1) -| (network);
		\draw[thick,rounded corners,-stealth] (network) |- node[fill=brown,align=center,inner sep=0,opacity=0.8](shell){\small Unix\\[-1ex] \small shell} (gui);

		\draw[thick,decorate,decoration={coil,aspect=0,amplitude=0.5mm,segment length=0.5cm}] (usb2) -| node[below right=1.5mm]{MAVlink} (uav) ;
		\draw[thick,decorate,decoration={coil,aspect=0,amplitude=0.5mm,segment length=0.5cm}] (mavproxy) -- (mav) |- (usb2) ;
		\draw[thick,-stealth] (gui) -- (script.west|-gui);
		\draw[thick,-stealth] (gui) -- (mavproxy);
		\draw[thick,decorate,decoration={coil,aspect=0,amplitude=0.5mm,segment length=0.3cm,pre length=5pt,post length=3pt},stealth-stealth] ($(mavproxy.north east)+(-3mm,-1mm)$) -- ($(script.south west)+(3mm,4mm)$);

		\draw[thick,decorate,decoration=zigzag] (sonar) -- (gpio);
		\draw[thick,decorate,decoration=zigzag,-stealth] (gpio) -| (script);

		
	\end{tikzpicture}
	\caption{Software architecture of the OCAS computer}
	\label{fig:software}
\end{figure}


As it can be deduced from Figure \ref{fig:software}, all the interaction from the Ground Control Station to the OCAS computer is performed through the Unix Shell, which only provides a command interface with the user.
Nevertheless, the SSH connection allows an optional X Window System protocol forwarding (as mentioned in Section \ref{sec:ocasintegration}); and a Graphical User Interface (GUI) will be developed in response to Requirement 8.1, allowing the execution of MAVproxy as well as the custom control scripts in order to enhance the intuitive operation of the system by the pilot.

\subsection{MAVproxy}

MAVproxy is an open-source Ground Control Station piece of software that is distributed as a Python application.
Thus, it can be run on any machine on which a Python distribution can be installed (virtually any operating system).

Its most significant difference compared with traditional GCS software such as Mission Planner or QGroundControl is that MAVproxy is built for the command line and does not need a graphical desktop environment to operate (although a small state window and map are also implemented), which means that it is the most adequate alternative for the operation of the UAV from the OCAS.
Furthermore, another decisive feature is the ability of MAVproxy of forwarding the MAVlink messages that are received from and sent to the UAV, allowing the possibility of operating as an intermediate software layer between the UAV and other GCS pieces of software.
That in particular is the chosen architecture for the OCAS in terms of communication between the UAV and the custom control scripts, which will be covered in Section \ref{sec:script}.

Concerning the present section, the setup of MAVproxy will be explained.

\begin{figure}[htbp]
	\centering
	\begin{tikzpicture}
		
		\newcommand{\drawuav}[1]{
			\node[#1](coord){};
			\coordinate[fill=gray,draw=gray,circle,minimum size=0.1cm,inner sep=0] (m1) at ($(coord)+(-0.4cm,0.4cm)$) {};
			\coordinate[fill=gray,draw=gray,circle,minimum size=0.1cm,inner sep=0] (m2) at ($(coord)+(0.4cm,0.4cm)$) {};
			\coordinate[fill=gray,draw=gray,circle,minimum size=0.1cm,inner sep=0] (m3) at ($(coord)+(0.4cm,-0.4cm)$) {};
			\coordinate[fill=gray,draw=gray,circle,minimum size=0.1cm,inner sep=0] (m4) at ($(coord)+(-0.4cm,-0.4cm)$) {};

			%Airframe
			\path[fill=gray,draw=gray,line width=0.7mm] (m1.center)
				.. controls (coord) .. (m2.center)
				.. controls (coord) .. (m3.center)
				.. controls (coord) .. (m4.center)
				.. controls (coord) .. (m1.center);
				

			%Propellers
			\path[fill=gray] (m1) arc (155:115:0.5cm) arc (335:295:0.5cm)
				arc (115:155:0.5cm) arc (295:335:0.5cm);
			\path[fill=gray] (m2) arc (65:25:0.5cm) arc (245:205:0.5cm)
				arc (25:65:0.5cm) arc (205:245:0.5cm);
			\path[fill=gray] (m3) arc (155:115:0.5cm) arc (335:295:0.5cm)
				arc (115:155:0.5cm) arc (295:335:0.5cm);
			\path[fill=gray] (m4) arc (65:25:0.5cm) arc (245:205:0.5cm)
				arc (25:65:0.5cm) arc (205:245:0.5cm);

			\node[yshift=-0.8cm](uav-label) at (coord) {\footnotesize UAV};

			\node[fit=(coord)(m1)(m2)(m3)(m4),inner sep=1.4mm]
				(uav) {};	%Group the UAV together
		}

		\tikzset{node distance=2cm}

		\drawuav{}
		\node[draw,fill=gray!50,minimum height=6mm,minimum width=8mm,right=3cm of uav](usb) {\includegraphics[width=5mm]{./figures/usb.png}};
			\node[below=0pt of usb]{\footnotesize /dev/ttyUSB0};
		\node[ellipse,draw,double,thick,inner sep=2mm,right=of usb](mavproxy){MAVproxy};
		\node[draw,rounded corners,below=1cm of mavproxy](window){State window};
		\node[draw,rounded corners,above=1cm of mavproxy](map){Map};
		\node[draw,semithick,minimum width=2cm,minimum height=1.5cm,above right=5cm of mavproxy,yshift=-3cm,align=center](script){Custom\\ control\\ script};
		\node[draw,semithick,minimum width=2cm,minimum height=1.5cm,below right=5cm of mavproxy,yshift=3cm,align=center](gcs){Other GCS\\ software};

		\draw[thick,stealth-stealth] (uav) -- node[above]{\footnotesize Serial comm.} (usb);
		\draw[thick,stealth-stealth] (usb) -- (mavproxy);
		\draw[thick,stealth-stealth] (mavproxy) -- node[above,sloped]{\footnotesize UDP: 127.0.0.1:14550} (script);
	\draw[thick,stealth-stealth] (mavproxy) -- node[below,sloped]{\footnotesize UDP: --.--.--.--:---} (gcs);
		\draw[-stealth] (mavproxy) -- (map);
		\draw[-stealth] (mavproxy) -- (window);


	\end{tikzpicture}
	\caption{MAVproxy setup}
	\label{fig:mavproxy}
\end{figure}


As previously mentioned and shown in Figure \ref{fig:mavproxy}, the connection from the UAV to the Raspberry Pi is done via a serial communication through USB cable.
The Ardupilot software utilises a baud rate of 115200 bd/s by default, which is an important parameter that needs to be specified to MAVproxy before connecting to the vehicle (else the connection will fail).
In addition, the address given by the OS to the USB port (/dev/ttyUSB0 in Figure \ref{fig:mavproxy}) has to be provided too. 

Finally, the redirection of MAVlink messages is done via the User Datagram Protocol (UDP), making use of the Internet Transport Layer.
However, a remote Internet connection is not needed, since both MAVproxy and the intended target (the custom control script) will be running on the same machine; hence the local IP address can be used (127.0.0.1) together with any available port of choice.
The optional second rerouting path is to be defined at the operator's discretion, and can be either a local or remote address.

The auxiliary Map and State windows are internally created by MAVproxy, using only Python function calls and libraries.

For completeness, the command to be executed in order to run the MAVproxy GCS software with the mentioned settings is:
\begin{lstlisting}[language=bash,basicstyle=\ttfamily\small,breaklines=true,breakatwhitespace=true,escapeinside={(*@}{@*)}]
	(*@\textbf{\$}@*) sudo ./mavproxy.py --master=/dev/ttyUSB0 --baud=115200 --out=127.0.0.1:14550 --out=127.0.0.1:14551
\end{lstlisting}

\subsection{The Python environment}


\section{The Python script}	\label{sec:script}




