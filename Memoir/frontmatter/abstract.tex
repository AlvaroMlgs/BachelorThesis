%
% File: abstract.tex
% Author: V?ctor Bre?a-Medina
% Description: Contains the text for thesis abstract
%
% UoB guidelines:
%
% Each copy must include an abstract or summary of the dissertation in not
% more than 300 words, on one side of A4, which should be single-spaced in a
% font size in the range 10 to 12. If the dissertation is in a language other
% than English, an abstract in that language and an abstract in English must
% be included.

\chapter*{Abstract}
\begin{SingleSpace}
\initial{T}he large growth that the civil Unmanned Aerial Vehicles (UAVs) market has experienced in the last decade is now triggering the urge of both professionals and enthusiasts to use this technology to perform tasks that would be more difficult to accomplish with their traditional procedures.
However, many times these tasks require precision flight and do not allow the slightest physical contact with the UAV. 
Currently, very qualified pilots are needed since there have not been significant advancements on on-board obstacle detection technologies, and manual control is still a must.

The main goal of this thesis is to develop an affordable Obstacle Alert and Collision Avoidance System (OCAS) that can be easily deployed to a wide range of UAVs.
The approach followed is to embark a series of ultrasonic rangefinders to continuously monitor the minimum distance of the vehicle with its surroundings.
The data provided by the sensors is then processed on an onboard computer, and control commands are sent to the main controller board in the case that an obstacle is detected and a possible collision identified.
The final result is an integrable payload subsystem that would improve the situational awareness capabilities of any UAV that integrates it, reducing the risk of collision with its surroundings.
\end{SingleSpace}
~\\
\begin{keywords}
	UAV, obstacle detection, collision avoidance, system integration, ultrasonic rangefinder, Ardupilot
\end{keywords}
\clearpage
